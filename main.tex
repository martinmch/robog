\documentclass{article}
\usepackage[a4paper,verbose]{geometry}
\usepackage[T1]{fontenc}
\usepackage[utf8]{inputenc}
\usepackage[danish]{babel}
\usepackage{parskip}
\usepackage{tgschola}
\usepackage{titlesec}
\usepackage[hidelinks]{hyperref}
\usepackage{graphicx}

\titleformat{\section}[display]
  {\centering\normalfont}{--- \thesection{} ---}{-0.2em}{\large}[]

\titleformat{\subsection}
  {\centering\normalfont}{\thesubsection.}{1em}{}[]

\title{Hvad enhver roer bør vide}
\author{Danske Studenters Roklub}
\date{juni 2015}
\begin{document}
\maketitle
%\newpage
\begin{center}
    \textbf{Behandling af rosportsmateriel, manøvrering m.m.}
\end{center}

Denne vejledning for roere og styrmænd i DSR er udarbejdet med det formål at
bevare romateriellet i den bedst mulige stand, således at alle aktive medlemmer
af DSR kan få den størst mulige glæde og udbytte af deres medlemskab.

\vfill


Redigeret af Christen Krogh, Pia Åbo Østergaard og Martin Christiansen.

\newpage

\tableofcontents


\section{Valg af båd}%

Reglerne for hvilke både, der må benyttes til en tur, fastlægges af bestyrelsen
og fremgår af opslag i bådhallen. Rovagten kan dog for hver enkelt tur træffe
ændrede bestemmelser herom f.eks. i forbindelse med reservationer. Uden for de
programsatte aktiviteter er det styrmandens ansvar at tage hensyn til evt.
reservations- og reparationsskiltning.

\section{Bådens bemanding}

Ingen båd må gå ud, med mindre den er fuldt bemandet. Mindst én af
mandskabet skal have styrmandsret. Det er forbudt at medtage passagerer.

\section{Styrmandens hverv}

Roere med almindelig styrmandsret har ret til at føre en robåd indenfor
DSR's daglige rofarvand. Dette strækker sig fra Sletten i nord, langs
kysten til Mosede/Dragør i syd. Vejen til Køgebugt går gennem slusen i
Københavns Havn. Styrmænd med almindelig styrmandsret må dog kun passere
opfyldningen i dagslys.

Styrmanden fører kommandoen og har ansvaret for hele turen, som regnes
fra det øjeblik, båden tages fra sin plads i bådhallen, til den atter er
sat på plads.

Styrmanden kan, lige meget hvor han sidder i båden, om nødvendigt tage
kommandoen. I både uden styrmandsplads fungerer den, der ror på 1.
pladsen, som aktuel styrmand, uden at dette dog fratager den ansvarlige
styrmand for ansvar.

Styrmanden har det fulde ansvar for mandskab og materiel, hvilket
omfatter
\begin{itemize}
    \item at sikre sig at mandskabet har de krævede rettigheder jf
        reglemet
    \item at reglementet og andre gældende bestemmelser overholdes
    \item at turen forløber på en forsvarlig og på en for roklubben og
        rosporten værdig måde
    \item at båden ikke kommer længere fra land, end at mandskabet kan
        bjærge sig selv og båden i land
    \item at båden med tilbehør behandles korrekt og med omhu
    \item at båden efter turen tømmes fuldstændigt for vand og rengøres
    \item at skader og mangler på materiellet forsøges repareret eller,
        hvis det ikke kan lade sig gøre, indføres i
        reparationsprotokollen
    \item at bådens tilbehør bevares i uskadt stand og efter turen
        anbringes på deres rette plads.
\end{itemize}

Såfremt nogen af turens deltagere skulle undlade at følge styrmandens
påbud, skal denne indberette det til bestyrelsen. Skulle nogen af roerne
mene, at styrmandens påbud er uretfærdigt eller urigtigt, må de ikke
undlade at følge dette, men kan efter turen klage til bestyrelsen.

Styrmanden skal altid, så vidt muligt, rette sig efter et ønske fra blot
én af roerne om at søge havn, gå tættere på kysten eller lignende,
således at ingen roere føler sig utrygge.

Selvom det ikke måtte være pligtigt, skal styrmanden respektere ethvert
ønske fra roerne om at medbringe redningsveste. Fremsættes et sådant
ønske, skal redningsveste medbringes til hele mandskabet. Det er
styrmandens pligt at sikre sig, at alle roere er bekendt med anvendelsen
af klubbens redningsveste.

\section{Klargøring af båd}

Før turen kontrolleres det, at årerne er i orden. Årelæderet på træårer
skal smøres ganske ubetydeligt.

Årerne bæres ud til pontonen, hvor de lægges for enden med årebladet ud
over kanten. Hvis det kan ske uden risiko for roere og årer, kan årerne
bæres ned to ad gangen. Ellers skal de bæres ned én ad gangen. Under ud-
og indbæring skal årerne bæres med bladet fremad for at undgå at det
beskadiges ved stød eller lignende.  Årer bæres altid i hoftehøjde – på
den måde mindskes risikoen for at skade andre roere.

Båden tages fra sin plads, og styrmanden sørger for, at ror, bådshage og
øsekar er på plads i båden samt at, stander, eventuelt flag og
redningsveste bliver anbragt i båden. Styrmanden kontrollerer, om
rorlinen og fanglinen er i forsvarlig stand, og sætter bundpropperne i.
Roerne efterser, om sæder, spændholter og bundbrædder er i orden.

Til en båd må ikke benyttes andet inventar - herunder årer - end det til
båden hørende. Mangler noget af en båds inventar, og kan det ikke skaffes
til veje, må båden ikke benyttes.

Findes en båd i snavset eller uordentlig stand, med manglende inventar
eller med vand i, skal dette straks rapporteres i roprotokollen. Såfremt
båden findes med eller der opstår en væsentlig skade, skal styrmanden
rapportere det i roprotokollens skadesafsnit. Er der ikke forud for turen
afgivet sådan rapport, hæfter styrmanden for bådens mangelfulde tilstand.

Overtages en båd med tilbehør, inden den er sat på plads, af et nyt hold,
hæfter styrmanden for det nye hold for bådens stand.

Inden turens begyndelse sikrer styrmanden, at mandskabet er skrevet ind i
roprotokollen, samt at afgangstidspunkt og turens bestemmelsessted er
anført.

\begin{figure}[htpb]
    \centering
    \includegraphics[width=0.8\linewidth]{figs/robaad}
    \caption{Diagram af en fireårers-inrigger. En toårers er magen til,
    blot med to færre sæder. Bemærk at \textit{stroke} altid sidder til
    bagbord.}
    \label{fig:robaad}
\end{figure}

\section{Båden sættes ud}

Båden køres på bådevognen ned ad slæbestedet og så langt ud i vandet som
muligt. Styrmanden løfter båden i stævnen og sætter den i vandet. Kølen
må ikke slæbes over vognens gjorde under udsætningen. Alternativt kan
bådvognen køres så langt ud, at båden bliver let og blidt kan trækkes af
bådvognen. Besætningen placerer årer, ror, stander og evt. flag.

Normalt er det ikke nødvendigt at fortøje båden. Det er tidskrævende og
er kun nødvendigt, hvis ikke styrmanden eller en af roerne hele tiden har
mulighed for at være ved båden og holde den ind til pontonen, indtil den
er klargjort.

Sørg for, at besætningen er klar til at gå på vandet før båden sættes ud,
og brug ikke unødig tid ved pontonen, hvis andre venter! Skulle et
besætningsmedlem være nødt til at træde af kan båden trækkes over til det
høje bolværk nord for pontonen, så andre ventende kan komme på vandet.

\section{Kommandoer og manøvrer}

\subsection{Kommandoens opbygning}

En kommando række er opdelt i tre dele:

\begin{enumerate}
    \item hvem kommandoen henvender sig til,
    \item hvad kommandoen går ud på, og
    \item iværksættelse af kommandoen.
\end{enumerate}

Eksempel: ``Bagbord \ldots til roning klar \ldots ro væk''.

Siges der intet under punkt 1, er det underforstået, at kommandoen gælder
alle roerne. Alle kommandoer skal afgives klart, tydeligt og roligt,
således at styrmanden sikrer sig, at alle roerne er parate til at udføre
den givne kommando. For hurtig, sjusket eller indforstået kommandogivning
medfører forvirring og mindsker styrmandens muligheder for at styre
båden.

\subsection{Der gås i båden}

Når årerne er anbragt i svirvlerne, stiller roerne sig ved kanten af
pontonen ud for deres pladser med front mod styrmanden, der holder båden
ind til pontonen.

Styrmanden kommanderer ``Ét og to klar til at gå om bord''; Roerne sætter
forsigtigt foden nærmest båden på trædepladen, midterlangremmen eller
bundbrættet. Samtidigt fattes med en hånd i hver side af båden for at
holde balancen; kroppens vægt holdes på pontonen.

På kommandoen ``(Ét og to) om bord'', sættes den anden fod ned i båden,
samtidig med at roerne sætter sig på sædet. Fødderne anbringes i
spændholtet.  Er der flere roere, kommanderes de derefter om bord på
samme måde to og to.  Styrmanden overvåger, at alt er i orden, og går
selv om bord.


\subsection{Der gås fra anlægsstedet}

Ligger båden så den kan komme fri ved enten at gå for eller agter,
kommanderes: ``Vi går agter/for''.  Styrmanden og de roere, der sidder
nærmest anlægsstedet, trækker derefter båden ud langs anlægsstedet med
hænder eller bådshage. Den sidste, der rører anlægsstedet, sætter fra,
således at båden kommer helt fri.

Hvis båden ikke på den måde kan komme ud, kommanderes: ``Stød fra'' Man
sætter fra med bådshage og årer, idet åreskafterne, der vender mod
anlægsstedet, benyttes til at skubbe fra med. Årebladet må aldrig bruges
til at skubbe fra med. Styrmanden hjælper selv med at sætte agterenden
fri med bådshagen.

\subsection{Roning og Skodning}

Roning starter med kommandoen: ``Til roning klar''.  Roerne kører frem i
fremstrækket og vender årebladet lodret.  Derefter kommanderes: ``Ro
(væk)''.  Roningen påbegyndes og fortsættes, indtil den afbrydes af en
anden kommando.  Tilsvarende kommanderes der til skodning med følgende
kommandoer: ``Til skodning klar ... skod (væk)''.

En manøvre (roning, skodning eller lignende) ophører ved kommandoen:
``Det er vel (evt. vel roet, velskoddet, ... )'' Denne kommando bør falde i
det øjeblik, åren forlader vandet. Åren bringes i en stilling vinkelret
ud for båden med årebladet over vandet i skivet (vandret) stilling.

`Det er vel' -stillingen er ikke en hvilestilling. Ønsker styrmanden at
tildele roerne et hvil, kommanderes: ``Åren på vandet'' Årebladet lægges
ned på vandet, og roeren kan nu gøre sig det bekvemt, indtil styrmanden
afbryder hvilet med `Det er vel'-kommandoen. For at undgå skader --- fx i
uroligt vand --- skal roerne hele tiden have fat på åren.

\subsection{Kommandoen `Fald ind`}

Kommandoen `Fald ind' bruges i situationer, hvor båden er i fart, og en
eller flere roer efter en `det er vel' skal til at ro med i samme rytme,
som øvrige. Dette er f.eks. tilfældet, når både vendes ved at den ene
side af både roer, mens den anden ikke roer og afventer kommando. Når
båden er ved at være vent kommanderes til roerne, der ikke roer, først:
``Klar til at falde ind'' efterfulgt af kommandoen: ``Fald ind'' Roerne,
der falder ind, roer med i samme rytme, som resten af båden.

\subsection{Båden standses}

Skal båden standses kommanderes: ``Det er vel ... Sæt i'' Roerne sænker
bladet i vandet i skivet tilstand, således at bladet kommer cirka 10 cm.
under vandoverfladen, og arme og ben strækkes for at kunne modstå
trykket.

Kommandoen `Sæt i' tager farten af båden. Skal båden standses helt
kommanderes derefter: ``Sæt hårdt i'' Roerne drejer årebladet til lodret
stilling med hulheden i bevægelsesretningen. Hvis båden bevæger sig
fremad, sættes hårdt i med årebladet vendt som i skoddetaget, og hvis
båden bevæger sig baglæns, sættes hårdt i med årebladet som i rotaget.
Arme og ben beholdes helt strakte under hele bevægelsen for at modstå det
ofte voldsomme tryk.

Er ordren `Sæt hårdt i' givet, fordi der er fare for at tørne mod et
bolværk, andre både eller lignende, kan der om nødvendigt nu kommanderes:
``Skod (væk) eller Ro (væk)'' alt efter bådens bevægelsesretning.
Skodde/Ro taget foretages straks uden at løfte bladet ud af vandet, da
åren jo står, som den skal.

Disse kommandoer afblæses med ``Det er vel''.

\subsection{Båden vendes}

Der kan vendes på flere måder:

Styrmanden kan lade roerne i den ene side ro/skodde, mens roerne i den
anden side sidder i `Det er vel'-stilling, sætter i eller sætter hårdt i.
Er båden i fart, kan styrmanden lade roerne i den ene side sætte i og
senere hårdt i, evt. mens den anden side fortsætter roningen/skodningen.
De to sider kan ro og skodde i modsat takt, dette sker efter kommandoen:
``Bagbord til skodning klar - styrbord skal ro i modsat takt.'' Efter
denne kommando gør roerne i bagbords side klar til skodning. Roerne i
styrbords side følger med tilbage på skinnerne og holder årene på samme
måde som de andre, dog i skivet stilling. Derefter kommanderes: ``Skod
(væk)''.

Skoddetaget fuldføres af roerne i bagbords side samtidigt med, at roerne
i styrbords side kører med frem på skinnerne. Når skoddetaget er
fuldført, er roerne i styrbords side klar til roning og gennemfører deres
rotag, mens roerne i bagbords side kører med tilbage på skinnerne til
skodning klar.

Hvis båden skal vendes styrbord over, startes med skodning i styrbord, og
bagbord følger med rotaget.

Ro- og skoddetagene må aldrig foretages samtidigt. Skodningen og roningen
i modsat takt fortsætter, indtil styrmanden kommanderer: ``Det er vel''.

\subsection{Roning/Skodning med kvart åre}

Hvis en passage er så smal, at der ikke er plads til årerne, kommanderes:
``Kvart åren''. Roerne trækker derpå åren indenbords således, at den
udenbords del af åren bliver noget kortere; hænderne flyttes tilsvarende,
således at roerne har en hånd på hver side af årelæderet. Roningen
fortsættes.

Hvis båden ikke er i bevægelse før en smal passage kommanderes: ``Til
roning med kvart åre klar Ro (væk)'', og ``(Til skodning med
kvart åre klar skod (væk)''

Roning/skodning med kvart åre afbrydes med kommandoen `Det er vel'. Når
åren atter skal ud i fuld længde, kommanderes: ``Åren ud''.

\subsection{Skift i båden}

Skal roerne skifte plads i båden, gøres det i reglen på den måde, at
1’eren kommer op som styrmand, mens den, der sidder på styrmandspladsen,
overtager tagåren, og de andre roere hver især flytter én plads frem i
båden.

In og halvout riggede både er forholdsvis stabile fartøjer. Man kan
bevæge sig rundt i båden, når det blot gøres med forsigtighed.

Den sikreste måde at skifte på er, at styrmanden kommanderer: ``Åren på
vandet''. Derefter skifter roerne plads en ad gangen ved at skræve over
åreskafterne. Der må kun trædes på de dertil egnede steder i båden. Når
en roer forlader sin åre, sørger den nærmestsiddende for, at den stadig
holdes på vandet.

Hvis vandet er ikke er for uroligt, kan der skiftes nemmere og hurtigere
ved, at styrmanden kommanderer: ``Åren langs''. Roerne fører da
åreskaftet forbi kroppen, som lænes tilbage. Dermed kommer åren til at
ligge langs båden med årebladet mod styrmandspladsen (agter). Når åren
ligger på denne måde, mister båden en del stabilitet, hvorfor roerne skal
holde kroppens tyngdepunkt så lavt som muligt, når de bevæger sig i
båden.  Skifter man med årerne langs, skal roerne, så snart de er på
deres nye plads, lægge årerne vinkelret på båden igen.

Alle i båden, bortset fra den, der aktuelt bevæger sig til en anden
plads, hjælper til med at holde balancen i forbindelse med skift i båden.

\subsection{Tillægninger}

Ved tillægninger er det oftest mest hensigtsmæssigt, at båden ikke har
alt for meget fart på. Dette opnås ved kommandoen: ``Til småt roning
klar.. Ro (væk)'' eller blot ``Småt roning''. Ved denne kommando ror
roerne videre som før, dog nu med mindre tryk på årerne, når de føres
igennem vandet.

Når der igen skal trækkes normalt kommanderes: `Normalt træk' eller, hvis
der kun har været roet småt i den ene side: `Lige træk'.

Ved tillægning ved ponton, hvor denne ikke er højere end bådens svirvler,
benyttes kommandoen: `Åren tværs' Åren trækkes da ind i båden, således at
den ikke rammer genstande på pontonen.

Hvis styrmanden har opdaget drivtømmer eller en anden forhindring, som
åren måske vil kunne berøre, kommanderes: ``Se til åren'' Roerne afgør
derefter selv, hvorvidt åren skal trækkes lidt ind, tværs eller evt.
langs. Såfremt forholdene tillader det --- fx ved tæt passage af en bøje
--- kan dette ske uden, at roningen afbrydes, hvorved bådens styrefart
opretholdes.

Ved tillægning ved højt bolværk eller andre særlige anlægspladser, hvor
pladsforholdene er ukendte eller ikke tillader, at åren forbliver i
svirvlerne, samt ved honnørafgivelse eller hilsen, kommanderes i god tid:
``Klar til at rejse åren.'' Ved denne kommando løsner roerne havelågen og
holder sig beredt til at rejse åren. Når der kommanderes: ``Rejs åren''
trækker roeren sin åre ind i båden så årelæderet (muffen) er fri af
swirvlen og griber med den udvendige hånd på den anden side af årelæderet
med et undergreb og løfter åren op i lodret stilling ved hjælp af begge
hænder. Åreskaftet anbringes på bundbrættet så tæt på kølen som muligt.
Derpå rettes årene ind og vendes, således at den hule side vender mod
styrmanden (agter). Åren holdes med den indvendige hånd øverst i højde
med skulderen og den anden hånd nederst ud for benene.

Når årerne skal på plads igen, kommanderes: ``Lad falde'' Hver roer
lægger forsigtigt åren ned på rælingen mellem svirvlen og styrmanden
(agten for svirvlen), retter svirvlen ind og lægger roligt åren på plads.
Forsøges åren lagt direkte ned i svirvlen, uden først at have hvilet på
rælingen, kan den blive unødigt ridset. Når åren er på plads, sætter
roerne sig i `Det er vel'-stillingen.

Outriggede både er så smalle, at man ikke må rejse årer i disse. Ved
honnørafgivelse og hilsen i sådanne både benyttes `Det er
vel'-stillingen, og der ses lige ud.

Ved tillægning ved højt bolværk eller andre særlige anlægspladser kan
man, når forholdene tillader det, bruge kommandoen: ``Åren langs'' Ved
denne kommando lægges årerne ind langs båden med årebladet agterefter og
løftet op af vandet. Denne kommando er kun mulig i både, hvor svirvlerne
har havelåger.

Årerne bringes ud igen med kommandoen: ``Åren ud'' Hvorefter roerne
drejer lægger åren på plads og går i `Det er vel'-stilling.

\subsection{Båden forlades}

Styrmanden er den første, der går i land.

Mens styrmanden holder båden ind til anlægsstedet, kommanderes: ``Ét og
to klar til at gå fra borde''. Disse rejser sig roligt, mens de støtter
sig til bådens rælinger. Foden nærmest pontonen sættes på pontonen, idet
kroppens vægt dog stadig holdes i båden. På kommandoen: ``(Ét og to) Fra
borde'' fører roerne vægten ind på pontonen og følger selv umiddelbart
efter, uden at skubbe båden væk.

Er der flere roere, kommanderes de fra båden på tilsvarende måde to og
to.

\section{Båden bliver ved anlægsplads}

Skal båden blive liggende i vandet, mens mandskabet er i land, skal
årerne placeres forsvarligt, helst i båden, og båden skal fortøjes. Dette
gøres ved, at fanglinerne bindes modsat hinanden enten begge ind mod
båden eller begge væk fra båden, og med så meget slæk, at dønninger kan
afbødes. Roret bindes, så det vender væk fra bolværket, eller tages helt
af. Eventuelle fendere fastgøres på en måde, så de ikke kan glide til
siden eller op i båden.

\section{Båden tages op}

Skal båden tages op, tages først årerne ud af svirvlerne og placeres et
forsvarligt sted. Dernæst tømmes båden for bagage. Ror, stander og
eventuelt flag tages af og lægges i båden.

Er der meget vand i båden, må den lænses med øse eller svamp, før den
tages op, da den ellers let bliver ødelagt af tyngden af vandet og i
øvrigt bliver unødigt tung at løfte.

Er der bådevogne, vælges en vogn, der passer til bådens størrelse To
striber på vognen må udelukkende bruges til 2 åres inriggere. En 4 åres
inrigger er for bred og skal derfor på en bådvogn med 4 striber. Er der
slæbested, køres vognen så langt ned ad dette som muligt. En af roerne
tager fat i stævnen og løfter forsigtigt båden op på vognen, som derefter
trækkes op på bådepladsen. Det er vigtigt at holde øje med, at bagenden
af båden også kommer til at ligge korrekt på bådvognen.

Skal båden bæres i land, gøres det ved at styrmanden løfter i bådens
stævn og trækker båden op. Så snart mandskabet kan få fat i
sidelangremmene, hjælper de til ved at følge bådens bevægelse og flytte
hænderne langs sidelangremmene, indtil de har fat så tæt på
styrmandspladsen som muligt, idet båden er tungest der.

Skal båden tages op på en flåde, ponton, bro eller lignende, må den først
lægges i retning vinkelret ud fra den kant, den skal tages op over.

\section{Landgang på åben strand}

Ved landgang på åben strand trækker roerne --- i tilfælde af høj sø
tillige styrmanden --- i god tid forud i bare ben. Båden ros vinkelret på
kysten ganske langsomt ind, idet dybden hele tiden pejles. Når der i
stille vejr er ca. en halv meter vand udfor 1- og 2 pladsen, træder nr. 1
og nr. 2 samtidig forsigtigt ud af båden, hver til sin side, og trækker
om fornødent båden videre frem, indtil nr. 3 og nr. 4 på samme måde kan
træde ud.

I tilfælde af sø må hele mandskabet, selv om det skulle blive vådt fra
øverst til nederst, ud af båden i så god tid, at der ikke er spor af fare
for, at den skal støde imod bunden.

Mens mindst én roer hele tiden bliver ude ved båden, bringes årerne på
land, og båden tømmes for al bagage og evt. bundvand, hvorefter den
løftes op på sædvanlig måde og bæres på land.

Når stranden forlades, sættes båden forsigtigt ud, vendes, så den ligger
parallelt med kysten, lastes med bagage, årer og roere, hvorpå styrmanden
vender båden, til vinkelret på kysten, sætter roret på og skubber båden
ud, mens han entre den.

Der er landgangsmuligheder på strandene ved Hellerup Havn,
Charlottenlund, Bellevue, Skodsborg, Vedbæk Havn og Rungsted Havn, samt
ved Amager Strand og Køge Strandpark. Når man går inden for de afmærkede
bade-/svømmeområder, skal man holde udgik efter badegæster.
`Langdistancesvømmere' ligger ofte helt ude ved bøjerne og kan være svære
at få øje på.

\section{Båden lægges op}

På langture eller på grund af høj sø, havari, tåge eller lignende kan det
være nødvendigt at lægge båden op for en periode. Kan dette ikke ske i en
anden roklub, må båden så vidt muligt anbringes på privat grund eller et
andet sted, hvortil der ikke er almindelig adgang. Kan båden ikke komme i
hus, lægges den med bunden opad på to korte lægter, brædder eller
lignende, der klodses op, så svirvlerne er fri af jorden.

Alt løst inventar bindes sammen og lægges under båden eller tages med
hjem. Afhængigt af forholdene kan båden fortøjes for og agter, så en
eventuel indtrædende storm ikke kan blæse båden bort. Bundpropperne tages
i forvaring, og årerne bringes så vidt muligt i hus.

Når båden er lagt på land, ligger den for holdets ansvar og risiko, og
holdet må gøre alt muligt for at sikre båden. Hvis den er lagt op af
ekstraordinære grunde (havari, uvejr eller lignende), må holdet sørge for
snarest at skaffe den hjem --- eventuelt efter samråd med bestyrelsen.

En udførlig redegørelse for oplægning og grunden hertil skal straks
afgives til bestyrelsen eller andre i klubben, som kan anføre i
roprotokollen, at båden ikke er i bådehallen pga. oplægning. Undladelse
heraf medfører erstatningsansvar ved eventuel eftersøgning.

\section{Materiellets aflevering}

Bundpropperne tages ud og lægges på deres pladser i båden ved siden af
styrmandssædet. Styrmanden sørger for, at stander og flag mm. lægges på
plads, og at turens længde og sluttidspunkt anføres i roprotokollen.

Mandskabet renser og aftørrer båden og årerne i henhold til klubbens
regler.

Såfremt der på en tur opstår eller opdages nogen skade på båd, årer eller
andet inventar, skal styrmanden straks efter turen forsøge at udbedre
skade. Er dette ikke muligt, meldes skaden i roprotokollens skadeafsnit
med detaljeret beskrivelse af skaden, præcis hvor på båden skaden er sket
og angivelse af skadesgrad (skade 1, 2 eller 3).

Ved større skader og især ved skader, der involverer andre parter skal
bestyrelsen ved rochef eller materielforvalter straks informeres med
detaljeret angivelse af hvordan og hvor skaden opstod.

Skader på materiellet, der ikke skyldes slid, ælde eller anden uforsætlig
hændelse, erstattes af den, der forvolder den. Havarier hæfter den for,
som er årsag hertil.

Såfremt det er sandsynligt, at en skade kan udvikle sig eller medføre
anden skade, skal båden hurtigt søge land, hvor en midlertidig reparation
foretages, hvis det er muligt. Ellers må båden lægges op.

Når båden skal på plads i bådehallen, skal den stilles samme sted, som
den blev taget. Skal båden løftes på plads, gøres det på den måde, at
styrmanden sammen med halvdelen af mandskabet tager agterenden og de
resterende tager stævnen.

Styrmanden kommanderer ``Fatte ... Løft ... Sæt''. Der gribes under
båden, og båden sættes forsigtigt på plads. Der kan med fordel søges
hjælp hos andre tilstedeværende medlemmer.

\section{Roning i høj sø}

Der må ikke ros ud i så høj sø, at der kan være fare for båden eller
mandskab. Skulle det blæse op undervejs, skal båden snarest søge land om
muligt havn. Under sådanne forhold må der være urokkelig ro og disciplin.

I høj sø går båden bedst, når den ligger i en skrå retning op imod søen.
Jo højere denne er, desto mindre vinkel med bølgerne bør båden lægges i
varierende imellem 10$^\circ$ og 45$^\circ$ således at den så vidt muligt
kan følge bølgebevægelsen uden at komme til at ride på bølgetoppene eller
hugge i søen.

I almindelighed kommer der 2 til 3 høje bølger lige efter hinanden. For
at disse ikke skal slå for hårdt imod båden eller give for meget vand
ind, lader man, lige inden de rammer, båden falde lidt af drejer den mere
parallelt med bølgerne ved hjælp af roret. Ved bådens drejning er det
ligesom den viger tilbage for slaget, og dette derved mildnes, men båden
må straks efter rettes op igen, f.eks. ved et par raske tag i læsiden.

Når søen er så høj, at båden må gå omtrent parallelt med bølgerne, er det
nødvendigt at lade den læ ræling ligge så lavt som muligt, men lige før
bølgetoppen passeres, må båden rettes op med begge rælinger i samme
højde, da der ellers tages vand ind om læ. Disse manøvrer udføres ved, at
mandskabet lægger kroppen mere eller mindre over i læsiden.

Styrmanden må stadig holde roerne ajour med, når der kommer en særlig høj
bølge, og må i det hele taget stadig instruere mandskabet.

Ved den beskrevne fremgangsmåde kan båden kun føres i de to retninger
under en forholdsvis bestemt vinkel op imod bølgerne, og da målet vel er
nogenlunde givet, vil det derfor i almindelighed blive nødvendigt at
krydse sig frem.

\section{Hvis båden fyldes eller kæntrer}

Hvis båden bordfyldes:

\begin{enumerate}
    \item Bevar roen
    \item Tag redningsveste på
    \item Om nødvendigt for at lette båden kan mandskabet helt eller
        delvist stige ud af båden.
    \item Tøm båden for vand. Mandskab op i den, når den kan bære
    \item Hold båden op mod søen
    \item Søg straks mod land
    \item Undersøg hvad der mangler af udstyr
    \item Underret myndighederne 1.
\end{enumerate}

Hvis båden kæntrer:

\begin{enumerate}
    \item Saml mandskabet, og bevar roen
    \item Tag redningsveste på
    \item Bind årene til fanglinerne
    \item Vend båden
    \item Tøm båden for vand, mandskab op i den, når den kan bære
    \item Hold båden op mod søen
    \item Søg straks mod land
    \item Undersøg hvad der mangler af udstyr
    \item Underret myndighederne
\end{enumerate}

(ovenstående er lånt fra DFfR's korttursstyrmandsmappe)

\section{Almindelige sikkerhedsregler}

På turen og ved landgang skal der vises al mulig forsigtighed, idet bl.a.
vind og sø tages i betragtning. Båden må ikke føres ind på så lavt vand,
at der er fare for, at den tager grunden, og navnlig må båden holdes
borte fra steder, hvor der kan tænkes at være sten, pæle, el.lign. i
nærheden af vandoverfladen.

Styrmanden må stadig holde udkig efter drivende genstande. Der må ikke
gås så tæt forbi nogen genstand, at der er fare for, at båd eller årer
kan støde imod.

I tilfælde af at der under turen opstår så stærk tåge, at orientering kan
blive vanskelig, skal der straks søges land; dette må ikke slippes af
syne, og der må gåes meget langsomt og forsigtigt frem.

Hvis der bliver høj sø, så der kan være fare for, at båden fyldes med
vand, må der straks søges land om muligt nærmeste havn.

\section{Færdselsregler på søen}

De internationale søvejsregler er søens færdselsregler. Søvejsreglerne
omfatter alle typer fartøjer, herunder robåde. Det er derfor nødvendigt,
at styrmanden har indgående kendskab til de vigtigste af disse.

Søvejsreglerne indeholder bl.a.\ regler om vigepligt, dvs.\ regler om,
hvilke typer af fartøjer, der skal gå af vejen for andre.

Robåde er ikke direkte omtalt i søvejsreglernes vigeregler. Grunden
hertil er, at begrebet ``tilfældets særegne omstændigheder'' for ofte
spiller ind, hvor en robåd møder et andet fartøj. Vigereglernes
hovedprincip er, at den stærke viger for den svagere. Anvendes dette
princip på robåde, betyder det, at skibe for motor og sejlskibe skal gå
af vejen for robåde med de modifikationer, som følger nedenfor:

Vores robåde er så hurtige og manøvreringsdygtige i forhold til
traditionelle robåde, at der kan opstå farlige situationer, fordi andre
fartøjer undervurderer vores fart. DSR har derfor vedtaget at uddanne
klubbens styrmænd med henblik på, at disse i god tid skal kunne forudse
en mulig farlig situations opståen, således at en sådan søges afværget
gennem manøvrering på et tidligt tidspunkt. Ved tilrettelæggelse af disse
manøvrer bør det sikres, at sejlskibe passeres på læsiden (den side, hvor
storsejlet føres), således at sejlskibet frit kan dreje til luvart (løbe
op i vinden).

Det hører til godt sømandskab i god tid tydeligt at markere, hvilken kurs
man vil holde, således at andre fartøjer kan manøvrere herefter. Såfremt
situationen imidlertid opstår, hvor der er en nærliggende fare for
kollision, må det haves for øje, at andre fartøjer, herunder sejlskibe,
vil tilrettelægge deres manøvre med henblik på, at disse fartøjer skal
vige, dvs., at disse fartøjer påregner, at robåden holder sig til
oprindelige fart og kurs.

Herudover kan der stadig forekomme situationer, hvor der er brug for
vigeregler. En robåd, der overhaler en anden båd, der er i bevægelse,
skal uanset hvilken type fartøj, der er tale om, gå af vejen – dvs.
passere til den side, hvor man er til mindst gene for den anden båd og
øvrig trafik. En robåd, der stævner mod en anden båd, skal holde til
højre. Er to både på skærende kurs, så der er fare for kollision, skal
den båd, der har den anden til sin højre side, vige.

Robåde skal i egen interesse holde sig uden for sejlruter og løb. Såfremt
passering alligevel er nødvendig, skal det ske på en måde, så det er til
mindst gene for den øvrige trafik. Passage af et løb skal således ske
vinkelret på løbet og må kun ske, hvis andre fartøjer ikke skal holde
tilbage eller ændre kurs.

I snævre farvande som kanaler, havneløb og lignende skal båden manøvreres
i højre side. Ved gennemsejling under broer med flere
gennemsejlingsåbninger skal båden tilsvarende passere åbningen til højre
for midten i gennemsejlingsretningen. I denne forbindelse skal det
indskærpes, at styrmanden holder sin robåd godt klar af et motordrevent
skibs skruevand, ligesom det skal undgås at komme i vejen for et skib,
der er ved at gå fra eller til kaj. En uventet manøvre med skruen kan
ellers bringe ens båd i en farlig situation.

De vigtigste lydsignaler til kommunikation af forestående manøvrering,
som bruges af skibe i sigte af hinanden, er som følger:

\begin{itemize}
    \item[1 kort stød] Jeg går styrbord (højre) om
    \item[2 korte stød] Jeg går bagbord (venstre) om
    \item[3 korte stød] Min maskine slår bak
    \item[5 korte stød] Hvad er det, du vil?
    \item[1 langt stød] Advarsel, her kommer jeg!
\end{itemize}

\section{Dagligt rofarvand}

Dagligt rofarvand strækker sig fra Sletten i nord, Dragør mod syd-øst,
Mosede i syd, samt kystnært. Den sydlige del af Amager er ikke en del af
dagligt rofarvand.

Man skal være særligt opmærksom på sten og stenrev følgende steder: ved
statuen af Knud Rasmussen, i bugten umiddelbart nord for Dragør, udenfor
sejlrenden i Østhavnen samt efter Rungsted Havn og i Nivåbugten

Generelt skal der roes udenfor badebøjerne ved Charlottenlund og Bellevue
m.m.\ --- ror man alligevel indenfor, grundet særlige sikkerhedsmæssige
omstændigheder, skal det foregå med nedsat fart og skærpet.

Der henstilles til, at der roes forbi Skovshoved Havn udenom bananen. Kun
hvis hensigten er at gå ind i havnen, roes der indenom bananen.

Derudover skal man være opmærksom på et overdækket beton kloakudløb ca.
50 m.\ nord for Hellerup Strand.

Ud for Knud Rasmussen er der ofte dykkere.

Bemærk, at arealet mellem Opfyldningen og Kronløbets vestlige kant er
udlagt som erhvervshavn.  Dette betyder, at vejen til Københavns Havn
eller Amager er blevet længere og mere udsat, idet man skal ganske langt
til havs for at komme forbi.

Derfor er disse restriktioner på passagen gældende indtil videre:

\begin{itemize}
    \item Redningsveste skal medbringes på alle ture, som går
        rundt om opfyldningen.

    \item Vejrudsigten skal tjekkes umiddelbart før passage af området,
        dvs.\ også inden hjemturen fra f.eks.\ Københavns Havn.

    \item Vær opmærksom på, at vindstyrken vil overraske, når man kommer
        ud på åbent hav.

    \item Styrmanden har en skærpet pligt til at oplyse besætningen om
        ruten inden man tager af sted.

    \item Hele besætningen skal give accept til turen inden den
        påbegyndes.

    \item Opfyldningen må kun passeres i svag vind
\end{itemize}

For korttursstyrmænd gælder desuden at:
\begin{itemize}
    \item Opfyldningsområdet må kun passeres i dagslys
    \item Vandtemperaturen skal være over 10 grader C (Scullerskilt skal
        være lukket)
\end{itemize}

Bestyrelsen indskærper at styrmanden skal foretage en realistisk
vurdering af besætningens evner inden turen påbegyndes, og anbefaler at
man følges mere end en båd. Langtursstyrmænd kan fortsat passere efter de
gældende regler for langture.

\section{Specielt om Københavns Havn}

Adgang til Københavns Havn skal ske via Lynetteløbet. Det er forbudt at
krydse Kronløbet i scullermateriel samt gig med en dunk som styrmand.

Generelt bør de generelle søfartsregler overholdes ifm.\ roning i
Københavns Havn, og det anbefales at udvise sund fornuft ved
undtagelsesvis fravigelse fra disse. For eksempel bør afmærkning af
større anlægsarbejder respekteres og en eventuel alternativ rute
planlægges i god tid. Københavns Havn kan på særlige tidspunkter være
særdeles trafikeret, og der bør altid roes med skærpet opmærksomhed i
havnen.

Der er adgang forbudt til Nyholmen og den inderste del af
Kalkbrænderihavnen. Sejlads med fritidsfartøjer i Yderhavnen, nord for
Toldboden og Elefanten/Batteriet Sixtus, skal jf.\ Havnereglementet foregå
i havnens østlige side, øst for de gule specialafmærkninger. Derudover
er dele af området ved Benzinøen spærret for robåde. Der er adgang til
Trekroner efter aftale.

Reglementet for Københavns Havn kan findes på
\href{http://www.byoghavn.dk/cphport.aspx} - se sejlinfo.

Det frarådes at ro ind i Nyhavn, især på tidspunkter, hvor
havnerundfarterne stadig er i rute. Vælger man alligevel at ro ind i
Nyhavn, må båden ikke lades uden opsyn.

Frederiksholms kanal er ensrettet, således at man skal ro højre om, dvs.
ind ved Bryghusbroen og ud ved Chr.\ IV's bro. Ensretningen skal
respekteres, men bemærk at den kun gælder robåde. I forbindelse med
metrobyggeriet ved Gl.\ Strand skal man være opmærksom på at der er endnu
mindre plads end ellers og at passage derfor kræver skærpet opmærksomhed.

For at kunne holde samme omløbsretning som (de fleste) turbåde bør man ro
gennem Christianshavns Kanal fra nord mod syd. Det er tilladt at lægge
til ved B\&W’s gamle hovedkontor i Christianshavns Kanal, men undgå at
benytte beboernes borde og bænke. Affald skal tages med.

Ror man bag om Holmen, skal man være opmærksom på, at turbådene normalt
sejler fra syd mod nord. Det kræver derfor skærpet opmærksomhed, når man
passerer hjørner og lignende steder med begrænset overblik, hvis man har
valgt at ro mod denne retning. I den lille kanal bag Arsenaløen ligger
der træer, cykler og lign.\ under vandoverfladen.

Der er et generelt forbud mod at gå i land både på selve Holmen og på
græsskråningerne mod øst. Man kan dog som regel få lov til at lægge til
ved plænen ved Rytmisk Musikkonservatorium i den nordlige del og ved
kajakklubben i den sydlige del af Holmen. Pas på slæbestedet ved
kajakklubben, da der går en stålskinne ud i vandet. Derudover er det
muligt at lægge til på siden af Kanalrundfartens `kilometer-bro'.

Tillægning ved havnerundfartens pladser bør kun ske, hvis man
øjeblikkeligt kan fjerne båden i tilfælde af, at en af rundfartens båd
dukker op. Generelt er det en meget dårlig idé at benytte disse
tillægningspladser, mens havnerundfartener i drift.

Vær opmærksom på Havnebusserne især ved afgang fra deres stoppesteder.
Her er bussernes skruevand særlig kraftigt.

Øst for Fisketorvet og Teglholmen er der meget lavt vand midt i
havneløbet. Det anbefales enten at følge sejlrenden på Amagersiden eller
at holde sig tæt mod Fisketorvet. Pas på badende!

Ved slusen i bunden af havnen er det ikke tilladt at færdes mellem
bropillerne, hvor strømmen dels kan være meget stærk, dels kan nogle af
portene synes åbne, men reelt kun være delvis åbne. Der er skiltet med
`Livsfare'.

Slusen må kun åbnes på bestemte tidspunkter. Information om
åbningstiderne se: http://kb-kbh.dk/slusen.htm.

Er det nødvendigt at måtte lægge båden op i havnen, må dette kun ske i
roklubberne i Københavns Havn. I lystbådehavnen ved Langelinie kan
havnefogeden, der gerne også er til stede om aftenen og i weekenderne,
hjælpe med at få låst båden inde. Længere inde i havnen ligger
roklubberne ARK og SAS på Islands Brygge og KR ved Tømmergraven i
Sydhavnen.

\section{Natteroning}

Roning efter mørkets frembrud må kun ske, når der i agterstavnen føres
klart lys, der er synligt horisonten rundt fra en afstand af mindst 2
sømil.

Der bør generelt roes med skærpet opmærksomhed og nedsat fart særlig ved
roning bag Holmen og i ind/udsejlingen af Lynetteløbet.

\section{Vinterroning}
    
Roning i perioden fra standerstrygning til standerhejsning må kun finde
sted i tidsrummet fra en halv time efter solopgang til en halv time før
solnedgang.

Kystlinjen følges nøje og der må ikke roes i is.

Roerne skal medbringe en godkendt redningsvest i perioden hvor
scullerskiltet er åbent og vandtemperaturen dermed er under 10 grader.
Hvorvidt vestene skal bæres, afgøres af styrmanden under hensyntagen til
roernes alder, erfaring og ikke mindst vejrforholdene.

\end{document}
